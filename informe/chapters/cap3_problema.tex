\chapter*{Introducción}
 \addcontentsline{toc}{chapter}{Introducción}

En la actualidad la sociedad se enfrenta a un cambio de paradigma en los sistemas de comunicación e información. Debido a la masificación de la tecnología a nivel mundial, el mejoramiento de las tecnologías de la información y la comunicación (TIC) se ha convertido en un factor clave en el desempeño productivo y el crecimiento económico e industrial. En Colombia, las empresas han aumentado significativamente el uso de las computadoras y el Internet. Se estima que para el año 2014 de 8.659 empresas el 99\% poseía computador y estaba conectada a Internet \cite{Comunicaciones2014}. El Ministerio de Tecnologías de la información y las comunicaciones ha invertido hasta \$373.993 millones de pesos Colombianos hasta marzo del 2014 solo en el proyecto de conectividad de alta velocidad, el cual busca que el 100\% de los municipios del país tengan acceso a Internet de alta velocidad \cite{Ardila2015}.
Los avances mencionados anteriormente han generado que las industrias modernas puedan almacenar grandes cantidades de datos en diferentes sistemas de información. Estos datos crecen rápidamente al ser recolectados por todo tipo de dispositivos, y son coleccionados por las industrias porque son una fuente valiosa de conocimiento, la cual puede ser usada para mejorar las decisiones relacionadas con la productividad. Sin embargo, actualmente el uso de estos datos históricos es limitado, ya que una gran cantidad de productos y datos quedan aislados y dispersos en los diferentes sistemas generando que las industrias sean ricas en datos, pero pobres en información \cite{Elovici2003}. De esta manera, la organización de estos datos y la búsqueda de conocimiento se convierte en un desafío para la minería de datos \cite{Hastie2009}.
La variabilidad de aplicaciones que generan el tráfico de datos en Internet es uno de los temas de investigación de las líneas de producto de software dinámicas, la autonomic computing y los web services \cite{Capilla2013}, ya que debido a dicha variabilidad los modelos de características industriales incluyen cientos de características derivadas de las preferencias de los clientes, lo cual los hace muy complejos y difíciles de configurar \cite{Asadi2014}; esto a su vez genera problemas y dificultades en la configuración y caracterización de los productos personalizados. Por estas razones surge la siguiente pregunta de investigación: ¿Cómo las técnicas de minería de datos pueden transformar los datos históricos de las compañías en modelos de características?
