\chapter{Resumen}
 %\addcontentsline{toc}{chapter}{Resumen}

La minería de datos se define como un conjunto de técnicas aplicables a diversas representaciones de datos que tienen como objetivo la asociación, predicción, clasificación, transformación, carga y extracción de información a partir de dichos datos. El reto en la minería de datos, más que el tamaño, formato o el almacenamiento de la información, es el análisis de la misma, es decir, la implementación de la técnica que se ajuste más a la situación que se desea evaluar \cite{Izenman2006}. Escenarios tan simples como el conocimiento de las tendencias en los consumidores \textemdash	para lo cual es preciso ajustar los datos a una curva mediante técnicas estadísticas y descriptivas\textemdash son muy llamativos para las industrias innovadoras que quieren ajustar sus productos a las necesidades de sus clientes, mejorando la capacidad productiva de su empresa, abaratando costos productivos y generando mayores ingresos.
Mediante el uso de la minería de datos y de diferentes técnicas de agrupamiento es posible recolectar la información necesaria para crear una gran gama de productos del mismo tipo, en donde varíen el color, el tamaño, la capacidad, la velocidad, etc. Como resultado se descubren las relaciones jerárquicas y las restricciones que se deben tener en cuenta a la hora de generar masivamente los productos que las compañías desean producir. Este proceso se encuentra enmarcado dentro de la etapa de análisis de la ingeniería de líneas de productos y, a partir del mismo, se conciben los modelos de características. Las líneas de producto tienen como paradigma compartir y administrar una gran cantidad de características con el objetivo de construir productos que satisfagan las necesidades específicas de un segmento o misión particular del mercado y que se desarrollan a partir de un conjunto común de activos básicos de una manera prescrita\cite{Nickel2015}. De esta forma, compañías del sector bancario\cite{Koutanaei2015}, tecnológico \cite{Lin2013} y manufacturero \cite{Bae2011} han hecho uso de las técnicas de minería de datos para el desarrollo de sus productos mediante la ingeniería de líneas de producto. 
En la actualidad, los modelos de características se generan de forma manual. Este trabajo se fundamenta en los conceptos de minería de datos e ingeniería de líneas de producto con el propósito de crear de forma autónoma los modelos de características; teniendo en cuenta la información arrojada por las técnicas de minería de datos y buscando que estos modelos sean de utilidad en el desarrollo de las líneas de productos.
\\
\textbf{Palabras Claves:}
Product line engineering, PMML, RIPS, Java, Phyton scikit-learn. 

