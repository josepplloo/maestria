\chapter{Abstract}
% \addcontentsline{toc}{chapter}{Abstract}

Data mining is defined as a set of techniques applicable to various data representations with aim association, prediction, classification, transformation, loading and extraction of information from such data. The challenge in data mining, rather than the size, format or storage of information, is the analysis of the same, that is, the implementation of the technique that best fits the situation to be evaluated \cite{Izenman2006}. Scenarios as simple as knowledge of consumer trends —which data must be adjusted to a curve by statistical and descriptive techniques— are striking for innovative industries that want to adjust their products to the needs of their customers, improving the productive capacity of the company, lowering production costs and generating higher revenues.
Using data mining and different clustering techniques it is possible to collect the information needed to create a wide range of products of the same type, where color, size, capacity, speed, etc. are varying. Thus, the hierarchical relationships and the restrictions that must be considered when massively generating products that companies wish to produce are discovered. This process is framed within the stage of analysis of the engineering of product lines and, from the same, the models of characteristics are conceived. The product lines have as paradigm to share and manage great number of characteristics with the aim of constructing products that satisfy the specific needs of a market segment or mission, and that are developed from a common set of basic assets in a prescribed way\cite{Nickel2015}. In this way, companies in the banking \cite{Koutanaei2015}, technology \cite{Lin2013} and manufacturing \cite{Bae2011} sectors have made use of data mining techniques for the development of their products through product lines engineering.
Currently, models of characteristic are generated manually. This work is based on the concepts of data mining and product line engineering with the purpose of creating autonomous models of characteristics; considering the information provided by the techniques of data mining and looking for these models to be useful in the development of the product lines.
\\
\textbf{Keywords:}
Product line engineering, PMML, RIPS, Java, Phyton scikit-learn. 