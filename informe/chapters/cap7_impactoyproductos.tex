\chapter{Productos}
 %\addcontentsline{toc}{chapter}{Productos}

Al finalizar esta investigación se tiene como producto un desarrollo de software, el cual generará modelos de características a partir de datos históricos; la industria aprovechará sus datos históricos para generar líneas de producto de una forma innovadora, es decir, modelará un diagrama de características usando las técnicas de minería de datos. En el futuro este será un reto con la presencia del Internet de las cosas, las aplicaciones que usan Big Data como fuente de datos y la Internet de las cosas, en donde cada producto tendrá una conexión a Internet (más datos que minar). Se espera que los productos no sean simples objetos con sensores informando sobre los cambios en el medio ambiente, sino que en realidad estos puedan contener el conocimiento y reaccionar adecuadamente dependiendo del contexto de negocio en el que se encuentren. Los modelos de características  le dan a la industria la posibilidad de ver gráficamente todos los posibles productos que se pueden generar. Con la incorporación de un método de minería de datos para la creación de modelos de características el afinamiento de las líneas de producto resultantes al usar este método dará a conocer productos con las características que los consumidores prefieren y podemos asegurar el impacto positivo en las industrias, el consumo y el medio ambiente cuando se puedan usar los productos derivados de esta investigación. 
En este proyecto se destaca que cada objetivo especifico tiene como propósito generar un producto, el primer objetivo especifico da como resultado el desarrollo del mapeo sistemático de la literatura (SMS), el segundo objetivo especifico produce una lista de los métodos y herramientas candidatas dentro de la ciencia de los datos, haciendo énfasis en la minería de datos con los algoritmos de clasificación y agrupamiento, el tercer objetivo especifico obtenemos un modelo de proceso y un modelo de producto, que muestre los elementos implicados y el flujo de trabajo que se debe seguir para convertir los datos en modelos de características, en el cuarto objetivo se desarrolla un método, este involucra el producto del objetivo anterior para generar un método que especifique las tecnologías y las actividades técnicas a desarrollar para la elaboración de los modelos de características, el quinto objetivo especifico desarrollamos un ejemplo utilizando las tecnologías encontradas en el objetivo cuatro. Los productos en este proyecto describen los pasos necesarios para desarrollar el objetivo general.  


