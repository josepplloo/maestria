\chapter{Conclusiones y Trabajos Futuros}
 \addcontentsline{toc}{chapter}{Conclusiones y Trabajos Futuros}

 \section{Conclusiones}



En estos momentos, una gran cantidad de datos es almacenada en bases de datos y almacenes de datos. Los datos crecen rápidamente porque la información se guarda usando periféricos de computadora, códigos de barras, sensores y sistemas biométricos. También una gran variedad de datos que se consideraban difíciles de manejar o que estaban aislados ahora tienen fines prácticos, estos datos son  grandes y complejos, con millones de registros y muchas variables. Además, diferentes agencias gubernamentales, instituciones educativas e industrias han acumulado estas grandes cantidades de datos, como ejemplo para el desarrollo de los objetivos del proyecto de investigación la Facultad Nacional de Salud Publica  pone a disposición del Grupo de Investigación de Ingeniería y Tecnologías de las Organizaciones y de la Sociedad (ITOS) los Registros Individuales de Prestación de Servicios de Salud (RIPS) que se definen como el conjunto de datos mínimos y basicos  que el Sistema General de Salud Social requiere para sus procesos cuya denominación, estructura y características se ha unificado y estandarizado para todas las entidades a que hace referencia el artículo segundo de la resolución 3374 de 2000 (las instituciones prestadoras de servicios de salud (IPS), de los profesionales independientes, o de los grupos de práctica profesional, las entidades administradoras de planes de beneficios y los organismos de dirección, vigilancia y control del SGSSS.) 
Al usar estudios sistemáticos  se redujo el sesgo en la investigación\cite{Petersen2015}, el resultado fue un estudio completo y replicable. Con los hallazgos realizados, se pudo seleccionar más fácilmente las técnicas de minería de datos que se aplicaron en la elaboración de los modelos de características. El SMS presento un conjunto de documentos caracterizados y categorizados en una gran cantidad de dimensiones y características.
 El entendimiento del negocio fue muy humilde en el sentido de que eran datos muy interpretables

\section{Trabajos futuros}